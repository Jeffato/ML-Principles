\documentclass{article}

% Language setting
\usepackage[english]{babel}

% Set page size and margins
\usepackage[letterpaper,top=2cm,bottom=2cm,left=3cm,right=3cm,marginparwidth=1.75cm]{geometry}

% Useful packages
\usepackage{amsmath}
\usepackage{graphicx}
\usepackage[colorlinks=true, allcolors=blue]{hyperref}

\title{CS461 HW 2}
\author{John Bailon}
\date{October 20, 2024}

\begin{document}
\maketitle

\noindent
Submission Files:

\section{MMSE Regression}
\subsection{Data Matrix}
From the given data points, we can construct the data matrix as follows:
\[
\Phi = \begin{bmatrix}
1 & 4 & 1 & 1\\
1 & 7 & 0 & 2\\
1 & 10 & 1 & 3\\
1 & 13 & 0 & 4
\end{bmatrix}
\]
\noindent
Where the first column corresponds to the bias term, second to x1, third to x2, and fourth to x3.

\subsection{Exact or Approximate Solution}

The normal equation (shown below) will give an MMSE-approximated solution.

\[
\Phi^t \Phi \begin{bmatrix}
w_0\\
w_1\\
w_2\\
w_3\\
\end{bmatrix} = \Phi^t \begin{bmatrix}
16\\
23\\
36\\
43
\end{bmatrix} 
\]

\noindent
The data matrix has a rank 3, indicating linear dependence. This system is under-determined and means that $\Vec{y}$ = [16, 23, 36, 43] does not lie in the same hyperplane as [1, 4, 1, 1], [1, 7, 0, 2], [1, 10, 1, 3], and [1, 13, 0, 4]

\subsection{Invertibility}
$\Phi^t \Phi$ is not invertible. Since $\Phi$ is a singular matrix, $\Phi^t \Phi$ must also be singular since its rank is at most the rank of $\Phi$, which in this case is 3. Instead, we can use the Moore-Penrose Pseudo Inverse. I calculated this using the pinv function in from numpy.linalg. Using this pseudo inverse, we can plug into the normal equation

\[
\Vec{w} = (\Phi^t \Phi)^+  \Phi^t 
\begin{bmatrix}
16\\
23\\
36\\
43
\end{bmatrix}
\]

\noindent
and we find the following weights.
\noindent
\[
\Vec{w} = [0, 3, 3, 1]
\]

\subsection{Another Proposed Model}

The original model given differs from mine. Looking first at the sample dataset, we see that it is relatively small at only 4 points. It is possible that this introduces sample bias, leading to the model learning incorrect patterns. Additionally, since we used a psuedo-inverse to solve for the weights, there is some error since it is only an approximation. Finally, there could be some intrinsic error present in our data collection. 

\subsection{New Data}
Adding the new data, we have the following data matrix.

\[
\Phi = \begin{bmatrix}
1 & 4 & 1 & 1\\
1 & 7 & 0 & 2\\
1 & 10 & 1 & 3\\
1 & 13 & 0 & 4\\
1 & 16 & 1 & 5\\
1 & 19 & 0 & 6\\
1 & 22 & 1 & 7\\
1 & 25 & 0 & 8\\
\end{bmatrix}
\]

\noindent
Similar to 1.2, this data matrix is still singular. Once again, we use the Moore-Penrose Pseudo Inverse. Repeating the same process as in 1.3 we find the following new weights. 

\[
\Vec{w} = [-0.0384, 2.9947, 3.0827, 1.01107]
\]
\noindent
This model still differs from the original model.[TODO: Finish chance to obtain the original model]

\subsection{Delete Data}
The original data matrix has the following linear dependency in columns 1, 2, and 4.

\[3 \begin{bmatrix}
    1 \\
    2 \\
    3 \\
    4 
\end{bmatrix} + \begin{bmatrix}
    1 \\
    1 \\
    1 \\
    1 
\end{bmatrix} = 
\begin{bmatrix}
    4 \\
    7 \\
    10 \\
    13
\end{bmatrix}
\]

\noindent
To restore full rank, we must delete one of these columns. Column 1 should not be deleted as it represents the bias term in weights. Either Column 2 or Column 4 can be deleted to ensure a unique solution, as the resulting matrix after their deletion results has a rank of 3.

\section{Lagrangian Functions and KKT Conditions}
\subsection{MMSE Objective Function}
\subsection{Lagrangian Function}
\subsection{Optimal Lagrangian Parameters}

\section{Learning Sinusoidal Functions}
\subsection{MMSE Regression}
\subsection{Ridge Regression}
\subsection{Plotted Models}
\subsection{Model Evaluation}
\subsection{Model w/o Regulation and Cross Validation}
\subsection{Solutions to Control Complexity}

\section{Eigenface}
\subsection{Spectral Decomposition}
\subsection{Image Representation}
\subsection{Eigenvalues and Human Faces}

\section{Estimate Art Creation- Deep CNN}
\subsection{Pre-processing}
\subsection{Training}
\subsection{Testing}

\end{document}